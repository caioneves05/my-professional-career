\documentclass[11pt,letterpaper]{article}

% Essential packages for formatting
\usepackage[utf8]{inputenc}
\usepackage[T1]{fontenc}
\usepackage{lmodern}
\usepackage[margin=1in]{geometry}
\usepackage{enumitem}
\usepackage[hidelinks]{hyperref}
\usepackage{xcolor}
\usepackage{titlesec}

% Font and layout settings for ATS readability
\renewcommand{\familydefault}{\sfdefault}
\setlength{\parindent}{0pt}
\setlength{\parskip}{6pt}

% Custom section headers for clarity
\titleformat{\section}{\large\bfseries}{\thesection}{1em}{}
\titlespacing*{\section}{0pt}{12pt}{6pt}

% Document starts here
\begin{document}

% Header with contact information
\begin{center}
    \textbf{\Large Engenheiro de Software Full Stack \& Desenvolvedor Mobile} \\
    \vspace{5pt}
    Guarulhos, São Paulo, Brasil \\
    +55 11 99962-9173 | +55 11 97525-2946 | ncaio037@gmail.com \\
    \href{https://www.linkedin.com/in/caioneves05}{linkedin.com/in/caioneves05} |
    \href{https://github.com/caioneves05}{github.com/caioneves05} |
    \href{http://caioneves.tech}{caioneves.tech}
\end{center}

% Summary section
\section{Resumo}
Sou Engenheiro de Software Full Stack e Desenvolvedor Mobile com forte experiência no design de sistemas distribuídos escaláveis, arquiteturas cloud-native e aplicativos móveis de alta performance. Domínio Java (Spring Boot), TypeScript, Node.js, React Native e serviços AWS (Lambda, ECS, ECR, SQS, SNS). Trabalho com DDD, Microsserviços, Clean Architecture, automação CI/CD e padrões de segurança (JWT, OAuth2). Entrego consistentemente melhorias mensuráveis em performance, confiabilidade e experiência do usuário.

% Technical Skills section
\section{Habilidades Técnicas}
\begin{itemize}[leftmargin=*]
    \item \textbf{Linguagens \& Mobile}: Java 17, TypeScript, JavaScript, Python, React Native, Expo.
    \item \textbf{Frontend}: Next.js, React.js, Angular, Tailwind CSS, HTML5, CSS3.
    \item \textbf{Backend}: Spring Boot (Security, Data, JPA), Node.js, NestJS, GraphQL, REST APIs, Microsserviços, Clean Architecture.
    \item \textbf{Bancos de Dados}: PostgreSQL (SQL), MongoDB (NoSQL), Redis, Kafka.
    \item \textbf{Cloud \& DevOps}: AWS (S3, Lambda, ECS, ECR, API Gateway, SQS, SNS), Docker, Kubernetes, Pipelines CI/CD, Git, Maven, Gradle, GitHub Actions.
    \item \textbf{Testes \& Qualidade}: JUnit, Jest, Testing Library, Testes de Integração, Testes Unitários, Automação de Testes.
    \item \textbf{Segurança \& Autenticação}: JWT, OAuth2, PKI (ICP-Brasil), Criptografia, Segurança de APIs.
\end{itemize}

% Professional Experience section
\section{Experiência Profissional}

\textbf{Desenvolvedor Full Stack - Pleno} \hfill Fev 2025 – Presente \\
\textit{RUK, Guarulhos, São Paulo, Brasil}
\begin{itemize}[leftmargin=*]
    \item Desenvolvi um microsserviço seguro de \textbf{Assinatura Digital} utilizando \textbf{Java}, Spring Boot e Bouncy Castle, garantindo conformidade total com a \textbf{ICP-Brasil} e assegurando validade jurídica no processamento automatizado de documentos.
    \item Criei pipelines escaláveis de processamento de áudio usando \textbf{Python}, \textbf{AWS Lambda} e S3, atingindo \textbf{95\% de precisão} em validações automáticas e reduzindo o tempo de revisão manual em 70\%.
    \item Otimizei a performance serverless ao reduzir a latência ponta a ponta em \textbf{65\%} (7s para 2.5s) por meio de ajustes avançados em AWS Lambda e otimização de imagens no \textbf{Amazon ECR}.
    \item Conteinerizei e orquestrei microsserviços utilizando \textbf{Docker} e \textbf{AWS ECS}, aumentando a confiabilidade do sistema em \textbf{30\%} e reduzindo o tempo de implantação em 40\%.
    \item Desenvolvi um sistema automatizado de validação de ponto usando \textbf{Node.js}, \textbf{TypeScript} e PostgreSQL, diminuindo erros de auditoria em \textbf{60\%} e reduzindo consideravelmente o esforço operacional.
\end{itemize}

\textbf{Desenvolvedor Full Stack} \hfill Set 2023 – Dez 2024 \\
\textit{Mobi Logística, Guarulhos, São Paulo, Brasil}
\begin{itemize}[leftmargin=*]
    \item Projetei sistemas backend escaláveis utilizando \textbf{NestJS}, \textbf{GraphQL} e PostgreSQL, garantindo consistência e alta disponibilidade com \textbf{99\% de uptime}.
    \item Desenvolvi e otimizei aplicativos mobile com \textbf{React Native} e Expo, aplicando técnicas avançadas de renderização que reduziram o tempo de carregamento das telas em \textbf{30\%}.
    \item Implementei uma arquitetura \textbf{offline-first} utilizando mecanismos de cache e armazenamento assíncrono, reduzindo o tempo de sincronização em \textbf{50\%} e assegurando operação em ambientes de baixa conectividade.
    \item Melhorei o desempenho de APIs ao otimizar consultas e estratégias de cache, reduzindo a latência em 25\% em endpoints críticos.
\end{itemize}

\textbf{Estagiário de Desenvolvimento de Software} \hfill Out 2022 – Ago 2023 \\
\textit{CreditHub, São Paulo, Brasil}
\begin{itemize}[leftmargin=*]
    \item Mantive e evoluí microsserviços distribuídos usando \textbf{Java} e \textbf{Spring Boot}, contribuindo para manter \textbf{99\%+ de uptime} nos sistemas principais.
    \item Desenvolvi e mantive \textbf{APIs REST} com Node.js e PostgreSQL, ampliando a escalabilidade, estabilidade e velocidade de desenvolvimento da plataforma.
    \item Implementei fluxos de testes automatizados com \textbf{Jest}, \textbf{JUnit} e Testes de Integração, aumentando a cobertura de testes e reduzindo defeitos em produção em \textbf{40\%}.
\end{itemize}

% Education section
\section{Formação Acadêmica}
\textbf{Tecnólogo em Análise e Desenvolvimento de Sistemas} \hfill Fev 2022 - Ago 2024 \\
\textit{Centro Universitário Eniac, Guarulhos, São Paulo, Brasil}

\textbf{Ensino Médio Completo} \hfill Dez 2021 \\
\textit{E.E. Homero Rubens de Sá, Guarulhos, São Paulo, Brasil}

\end{document}

\documentclass[11pt,letterpaper]{article}

% Incluindo pacotes essenciais para formatação
\usepackage[utf8]{inputenc}
\usepackage[T1]{fontenc}
\usepackage{lmodern}
\usepackage[margin=1in]{geometry}
\usepackage{enumitem}
\usepackage[hidelinks]{hyperref}
\usepackage{xcolor}
\usepackage{titlesec}

% Configurando fontes e layout para legibilidade em ATS
\renewcommand{\familydefault}{\sfdefault}
\setlength{\parindent}{0pt}
\setlength{\parskip}{6pt}

% Personalizando cabeçalhos de seção para clareza
\titleformat{\section}{\large\bfseries}{\thesection}{1em}{}
\titlespacing*{\section}{0pt}{12pt}{6pt}

% Definindo um ambiente itemize personalizado para conquistas
\newlist{achievements}{itemize}{1}
\setlist[achievements]{label=$\cdot$, leftmargin=*, itemsep=2pt}

% Início do documento
\begin{document}

% Cabeçalho com informações de contato
\begin{center}
    \textbf{\Large Engenheiro de Software Full Stack \& Desenvolvedor Mobile} \\
    \vspace{5pt}
    Guarulhos, São Paulo, Brasil \\
    +55 11 99962-9173 | +55 11 97525-2946 | ncaio037@gmail.com \\
    \href{https://www.linkedin.com/in/caioneves05}{linkedin.com/in/caioneves05} |
    \href{https://github.com/caioneves05}{github.com/caioneves05} |
    \href{http://caioneves.tech}{caioneves.tech}
\end{center}

% Seção de Resumo
\section{Resumo}
Sou Engenheiro de Software Full Stack e Desenvolvedor Mobile, com experiência no desenvolvimento de aplicações web e mobile escaláveis, focadas em desempenho e boa experiência do usuário. Atuo com TypeScript, JavaScript, React Native, Next.js, NestJS e GraphQL, além de backend em Java (Spring Boot, Spring Security, Spring Data JPA, Spring MVC, Mockito), Node.js, MongoDB, PostgreSQL e Kafka. Tenho conhecimentos em infraestrutura de nuvem utilizando serviços da AWS, como S3, Lambda, ECS, ECR, SQS e SNS.

No dia a dia, aplico boas práticas de desenvolvimento, como Domain-Driven Design (DDD), Clean Code e integração contínua (CI/CD), buscando entregar soluções confiáveis e de fácil manutenção. Possuo experiência com controle de versão usando Git e colaboração em times ágeis. Gosto de atuar em projetos que proponham desafios, onde seja possível otimizar processos, melhorar desempenho e criar soluções que gerem valor real para o usuário e para o negócio.

% Seção de Competências Técnicas
\section{Competências Técnicas}
\begin{itemize}[leftmargin=*]
    \item \textbf{Frontend \& Mobile}: TypeScript, JavaScript, React Native, Nextjs, Angular, Tailwind CSS, Expo
    \item \textbf{Backend \& APIs}: Java, Spring Boot, Spring Security, Spring Data JPA, Spring MVC, Mockito,  Spring Data, Spring Cloud Gateway / Hibernate,  Node.js, NestJS, GraphQL, MongoDB, PostgreSQL, Kafka, Jest
    \item \textbf{Nuvem \& DevOps}: AWS (S3, Lambda, ECS, ECR, SQS, SNS), Docker, Git, pipelines CI/CD
\end{itemize}

% Seção de Experiência Profissional
\section{Experiência Profissional}

\textbf{Desenvolvedor Full Stack - Pleno} \hfill Fev 2025 – Atual \\
\textit{RUK, Guarulhos, São Paulo, Brasil}
\begin{achievements}
    \item Desenvolvi pipelines otimizados de processamento de áudio usando Python e AWS Lambda, integrando com S3 e ECR para atender aos objetivos de negócios, alcançando 95\% de precisão em validações automatizadas.
    \item Criei soluções para validação de registros de ponto de funcionários por meio de análise de áudio, utilizando TypeScript e Nodejs, automatizando processos e reduzindo erros manuais em 60\%.
    \item Implementei microsserviços conteinerizados com Docker e AWS ECS, otimizando custos de infraestrutura e aumentando a confiabilidade do sistema em 30\%.
    \item Reduzi o tempo de processamento de áudio de 7 segundos para 2,5 segundos ao implantar AWS Lambda com imagens Docker no Amazon ECR, utilizando um modelo econômico de pagamento por solicitação.
\end{achievements}

\textbf{Desenvolvedor Full Stack} \hfill Set 2023 – Dez 2024 \\
\textit{Mobi Logística, Guarulhos, São Paulo, Brasil}
\begin{achievements}
    \item Construí sistemas escaláveis usando NestJS, GraphQL, MongoDB e PostgreSQL, garantindo disponibilidade e manipulação eficiente de consultas de dados em larga escala.
    \item Construí e otimizei aplicações móveis com Typescript, React Native, Tailwind CSS e Expo, integrando gráficos interativos para reduzir o tempo de carregamento de visualizações em 30\%.
    \item Aprimorei uma aplicação de entrega logística com abordagem offline-first utilizando estados globais e cache assíncrono com React Native e TypeScript, reduzindo o tempo de sincronização de dados em 50\% e possibilitando operação contínua em áreas de baixa conectividade.
\end{achievements}

\textbf{Desenvolvedor de Software - Estágio} \hfill Out 2022 – Ago 2023 \\
\textit{CreditHub, São Paulo, São Paulo, Brasil}
\begin{achievements}
    \item Apoiei a plataforma CreditHub com Java e Spring Boot, melhorando a estabilidade do serviço e mantendo 99\% de tempo de atividade.
    \item Desenvolvi e documentei novos microsserviços usando Node.js e PostgreSQL, aprimorando a escalabilidade e a manutenibilidade da plataforma.
    \item Resolvi problemas operacionais diários, utilizando Jest para testes unitários para garantir a qualidade robusta do código e reduzir a incidência de bugs em 40\%.
\end{achievements}

% Seção de Educação
\section{Educação}
\textbf{Análise e Desenvolvimento de Sistemas} \hfill Fev 2022 - Ago 2024 \\
\textit{Centro Universitário Eniac, Guarulhos, São Paulo, Brasil}

\textbf{Ensino Médio} \hfill Dez 2021 \\
\textit{E.E. Homero Rubens de Sá, Guarulhos, São Paulo, Brasil}

\end{document}
